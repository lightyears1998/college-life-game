% !TEX program = xelatex

\documentclass{ctexart}
\usepackage{ctex}
\usepackage[a4paper, left=20mm, right=20mm, top=30mm, bottom=30mm]{geometry}
\usepackage{graphicx}
\graphicspath{ {./images/} }
\usepackage{amsmath}
\usepackage{hyperref}
\hypersetup{
  colorlinks=true,
  linkcolor=black
}
\usepackage{xcolor}
\usepackage{listings}

% listings宏包配置,引用自:
% 怎么在 LaTeX 中排版 Python 代码? - 孟晨的回答 - 知乎
% https://www.zhihu.com/question/65508676/answer/232267619
\lstdefinestyle{lfonts}{
  basicstyle   = \footnotesize\ttfamily,
  stringstyle  = \color{purple},
  keywordstyle = \color{blue!60!black}\bfseries,
  commentstyle = \color{olive}\scshape,
}
\lstdefinestyle{lnumbers}{
  numbers     = left,
  numberstyle = \tiny,
  numbersep   = 1em,
  firstnumber = 1,
  stepnumber  = 1,
}
\lstdefinestyle{llayout}{
  breaklines       = true,
  tabsize          = 2,
  columns          = flexible,
}
\lstdefinestyle{lgeometry}{
  xleftmargin      = 20pt,
  xrightmargin     = 0pt,
  frame            = tb,
  framesep         = \fboxsep,
  framexleftmargin = 20pt,
}
\lstdefinestyle{lgeneral}{
  style = lfonts,
  style = lnumbers,
  style = llayout,
  style = lgeometry,
}
\lstdefinestyle{python}{
    language = {Python},
    style    = lgeneral,
}

\author{
	冯国蕴 \and 林欣煜 \and 马詠汛 \and 谢金宏
}
\title{使用细胞自动机模拟简单生态系统}
\date{2019年12月21日}

\begin{document}

\maketitle

\begin{abstract}
在本次课题中,我们小组使用Python语言实现了基本的细胞自动机。并在已有的细胞自动机的规则基础上进行扩展,尝试对具有氧气、生产者和消费者三要素的生态系统进行模拟,并将模拟的结果使用图像化的方式表现出来。
\end{abstract}

\tableofcontents

\section{细胞自动机与康威生命游戏}

细胞自动机\footnote{Cellular automaton \url{https://en.wikipedia.org/wiki/Cellular_automaton}}最早由冯·诺依曼在1950年代为模拟生物细胞的自我复制而提出,起初未受到科学界的广泛关注。后因约翰·何顿·康威设计了生命游戏\footnote{Conway's Game of Life \url{https://en.wikipedia.org/wiki/Conway's_Game_of_Life}}而闻名于世。本节将简要介绍细胞自动机与康威生命游戏。

\subsection{细胞自动机}

我们可以在二维的格状棋盘上实现细胞自动机。在这个二维的空间上,棋盘中每个格子内细胞的状态是有限的,细胞下一时刻的状态由该细胞的邻居在当前的时刻的状态决定。棋盘内的所有细胞遵守相同的演化规则。

一个细胞自动机一般具有以下特点:

\begin{itemize}
  \item \textbf{平行计算 }每个细胞个体的状态都同步地进行改变。
  \item \textbf{局部性 }细胞的状态只受相邻的细胞的影响。
  \item \textbf{一致性 }所有的细胞受到相同的规则约束。
\end{itemize}

\subsection{康威生命游戏}

康威生命游戏符合细胞自动机的特点,其规则定义如下:

\begin{itemize}
  \item 每种细胞有“存活”和“死亡”两种状态。
  \item 当细胞周围的存活细胞数量等于3个时,细胞变为存活状态。
  \item 当细胞周围的存活细胞数量少于等于1个或大于等于4个时,细胞变为死亡状态。(模拟细胞过于孤独或环境过于拥挤。)
  \item 其他情况下细胞状态不变。
\end{itemize}

我们小组使用Python实现了经典的康威生命游戏(\underline{1.basic.py})。程序先随机地给定棋盘内细胞初始状态,然后按照康威生命游戏规则演化指定的代数,给出最终的结果。

\begin{figure}[h]
  \centering
  \includegraphics[scale=0.75]{cellular-automation.png}
  \caption{按康威生命游戏规则进行一次迭代}
  \label{fig:cellular-automation}
\end{figure}

图\ref{fig:cellular-automation}是随机产生初始状态后按康威规则演化一代的结果。图中黑色块表示活细胞,白色块表示空位或死细胞。

\subsection{镜像边界}

细胞自动机理论上具有无穷大的“棋盘”作为细胞演化的空间,但计算机的内存空间是有限的,故无限大的棋盘不能实现,实际运行中的棋盘往往是规模为$n$的矩形棋盘。有限的空间就存在边界问题,边界上的细胞因此需要特殊处理。

在\underline{1.basic.py}中,边界上的细胞不参与演化,它们总是保持死细胞的状态。这是处理边界问题的一种方法,即边界上的细胞取得定值,在演化过程中保持不变。

另一种处理边界细胞问题的方法是在\underline{1.basic-mirror-edge.py}中实现的镜像边界。如图\ref{fig:mirror-edge-example}所示,编号为1的细胞,它的左上角邻居、左邻居和右邻居分别被镜像地设置为9号、3号和7号细胞。

\begin{figure}[h]
  \centering
  \includegraphics[scale=0.75]{mirror-edge-example.png}
  \caption{镜像边界的一个实例}
  \label{fig:mirror-edge-example}
\end{figure}

一般地,对于采用镜像边界的$SIZE$大小的棋盘,格点位置为$(i, j)$的细胞,它的邻居可以通过如下程序遍历:

\begin{lstlisting}[style = python]
for i, j in CELLS:
  for dx in range(-1, 2):
    for dy in range(-1, 2):
      neighbor = board.iat[(i + dx + SIZE) % SIZE, (j + dy + SIZE) % SIZE]
\end{lstlisting}

如无特殊说明,本文后续的模型均采用镜像边界的处理方法。

\subsection{康威生命游戏的几个经典图案}

我们在程序中实现了康威生命游戏的几个经典图形。包括一些永远保持“静止”不发生变化的图像(如图\ref{fig:classic-fixed})和一些在有限的图形中顺序切换,周而复始的图形(如图\ref{fig:classic-vibrating})。

\begin{figure}[h]
  \centering
  \includegraphics[scale=0.75]{classic-fixed.png}
  \caption{永远保持静止的图形}
  \label{fig:classic-fixed}
\end{figure}

\begin{figure}[h]
  \centering
  \includegraphics[width=\textwidth]{classic-vibrating.png}
  \caption{周而复始的图形}
  \label{fig:classic-vibrating}
\end{figure}

\section{氧气模型}

氧气是模拟生态系统中的要素之一,本节将介绍我们小组定义的氧气模型。

\subsection{扩散规则}

在现实生活中,氧气可以在自由空间内扩散,最终平均扩散到空间各处,各处的氧气浓度一致。因此要模拟氧气,最重要的是模拟氧气随着时间的推移而扩散到自由空间内的性质。

我们使用迭代次数的增加表示时间的推移。棋盘上每个格子的体积一定,因此一个格子上所含有的氧气的量可以用氧气浓度的数值进行表示和参与运算。规定格子上氧气的浓度可以在某次迭代中超过1,但不能在连续的几次迭代中持续超过1。

对于一个格子而言,格子在下一轮氧气的浓度为上一轮氧气浓度减去本轮扩散到其他格子的氧气浓度,再加上本轮其他格子扩散到当前格子的氧气浓度。

规定一个格子$c$在一轮扩散中氧气的总扩散量为上轮氧气浓度$last_c$的一半,对于与其相邻的包括它自身在内的9个格子$neighbor(c)$,每个格子$i$从$c$中获取的氧气的扩散量$shared_{c, i}$由以下公式进行计算:

\begin{equation}
  \label{equ:share-oxygen}
  shared_{c, i} = \frac{1}{2} \cdot last_c \cdot \frac{1 - min(1, last_i)}{\sum (1 - min(1, last_i))}, i \in neighbor(c)
\end{equation}

上述公式的物理意义是,在氧气扩散总量一定的前提下,格子的原有浓度越小,所分得的氧气浓度就越高。公式中分子和分母位置的$1 - min(1, last_i)$的含义是,氧气不向氧气浓度大于等于1的格子扩散。

那么对于一个格子$c$,其本轮的氧气浓度由以下公式进行计算:

\begin{equation}
  \label{equ:gathering-oxygen}
  current_c = \frac{1}{2} \cdot last_c + shared_{i, c} , i \in neighbor(c)
\end{equation}

按上面的公式\ref{equ:share-oxygen}和公式\ref{equ:gathering-oxygen}进行的计算是遵循物质守恒定律的。

对于棋盘边界上的格子,如果采用定值方法处理边界,则氧气浓度应该设置为1,且边界格子不参与氧气扩散量的计算。由公式\ref{equ:share-oxygen}知,边界格子亦不会吸收氧气。

如果采用镜像边界,则边界格子与非边界格子使用同样的规则进行迭代。

\subsection{氧气模型的实现}

我们在程序\underline{2.oxygen.py}中实现了上一小节中的氧气模型。

下面的热力图中,蓝色表示氧气。格子中的氧气密度越高,格子上蓝色的程度就越深。如图,棋盘的初始状态随机给出,随后按照我们定义的氧气规则进行演化。

如图,我们还模拟了棋盘中氧气从一处或几处浓度峰点向外自由扩散的情况。

从图像上看,我们的氧气模型一定程度上成功地模拟了现实世界的情况。

\section{光照和生产者模型}

生态系统可分为生物部分和非生物部分。在上一节中,我们在非生物部分中选择了氧气这样要素进行了模拟;本节中,我们将模拟生物部分的基础——生产者。

\subsection{氧气生产规则}

以下是光合作用的公式。

\begin{equation}
  12H_2O + 6CO_2 \overset{阳光}{\rightarrow} C_6H_{12}O_6 + 6O_2 + 6H_2O
\end{equation}

从光合作用的公式来看,光合作用的原料是水和二氧化碳,条件是阳光,最终产物是葡萄糖、水和氧气。我们模拟的生态系统中,缺乏

考虑生产者产生氧气时,我们忽略生产者的呼吸作用,只考虑净光合作用的影响。

单位格子上的,与氧气浓度类似,我们定义生产者的种群密度$density$。

图为程序\underline{}模拟的生产者生产氧气的过程。

\subsection{种群密度的增长规则}

\subsection{繁衍规则}

\subsection{生产者模型的实现}

\section{消费者模型}

\subsection{呼吸和消费规则}

\subsection{移动规则}

\subsection{繁衍规则}

\section{不同参数下的模拟生态系统}

不同阳光

消费者的不同移动倾向

\section{结语}

其实我们最初的想法没有那么简单。最初的想法是在建立氧气、生产者和消费者模型的基础上,结合遗传算法,模拟生物选择和进化的过程。但时间不够,只好作罢。

我们在对细胞自动机进行“拓展”的同时,其实已经破坏了细胞自动机的一些性质。

\end{document}
